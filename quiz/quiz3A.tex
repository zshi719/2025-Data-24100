\documentclass[11pt]{article}
\usepackage{fullpage, fancyhdr, ifthen, amssymb, amsfonts, amsmath, booktabs}
\usepackage{tabularx}
\usepackage{multicol}
\usepackage[margin=0.25in]{geometry}
\usepackage{upquote}
\usepackage{forest}
\usepackage{ulem}
\usepackage{colortbl}
\pagestyle{empty}

\begin{document}
%\begin{tabularx}{\textwidth}{l|X}
%\textbf{Quiz 3A} &   \textbf{Name: } \\
%\hline
%\end{tabularx}

\begin{table}[h]
\centering
\arrayrulecolor{black!10} % Very light table outline
\begin{tabular}{|p{\textwidth}|}
\hline
\begin{minipage}[t]{\textwidth}
\vspace{0pt}  % This is crucial for top alignment
\noindent \textbf{Instructions:} Please answer the following questions making sure to write your answers legibly on the second page. The code below is a segment of a python flask application that we will make changes to.
\begin{itemize}
\item Since we are writing python code please be careful about spacing / indent. Leave enough room that it is obvious when you want things indented. 
\item Make sure to write legibly.
\end{itemize}
\end{minipage} 
\\
\hline
\end{tabular}
\end{table}

{\color{lightgray}\hrule}
\begin{verbatim}
	
from flask import (
    Flask,
    request,
    Response
)

from data_processor import (
  api_status_code,
  api_data_interface
)

app = Flask(__name__)

# WHERE YOUR ANSWER WILL BE

@app.route('/api/v1/data', methods=['GET'])
def return_data_from_api_interface():
    '''
    Main Data route for our flask app
    '''
    text_data = api_data_interface()
    return Response(data, status=200,
                    headers={'Content-type': 'text/html'})


if __name__ == '__main__':
    app.run(host='0.0.0.0', port=5000)
\end{verbatim}
{\color{lightgray}\hrule}

\vspace{1cm}
\noindent We also have the following makefile. Note that the docker image, via a \texttt{CMD} statement starts the flask app as the default command.

\begin{verbatim}
IMAGE_NAME=flask_app

.PHONY=build run_flask 

build:
    docker build . -t $(IMAGE_NAME)

run_flask: build
    docker run $(IMAGE_NAME)

\end{verbatim}
{\color{lightgray}\hrule}

\clearpage

\begin{tabularx}{\textwidth}{l|X}
\textbf{Quiz 3A} &   \textbf{Name: } \\
\hline
\end{tabularx}

\vspace{1cm}

\begin{enumerate}
	\item Our flask server puts logs in the following location \texttt{/logs/flask.log} (on the host NOT inside the container). When there is an error in our log file the phrase \texttt{FLASKERROR} appears on the line. Please write make commands (you can ignore modifying the \texttt{.PHONY} line) which do the following:
	\begin{enumerate}
		\item Write a make command that returns the last 10 errors from the log (Name it \texttt{last\_10\_errors}).
		\vspace{3.5cm}
		\item Write a make command that returns the count of the number of lines in the log  that have errors (name it \texttt{total\_errors}). 
	\end{enumerate}
	
	
	\vspace{3.5cm}
	
	 
	\item We want to write a route (\texttt{/api/v1/status}) which responds to a \texttt{GET} request and returns a status code which tells us if the API is working or not. The imported function \texttt{api\_status\_code} returns 0 if the API is down, 1 if the API is working and 2 if the API is having issues (but still running). 
		\begin{itemize}
			\item Please write a complete route (including all required decorators, etc) that can be placed into the file.
			\item It should return status code 200 if the API is working (1)
			\item It should return a status code 500 if the API is down (0)
			\item It should return a 503 if the API is having issues (2)
			\item Nothing needs to be in the body or headers. 
			\item No need for comments or doc strings.
		\end{itemize}
\end{enumerate}


\end{document}