\documentclass[11pt]{article}
\usepackage{fullpage, fancyhdr, ifthen, amssymb, amsfonts, amsmath, booktabs}
\usepackage{tabularx}
\usepackage{multicol}
\usepackage[margin=0.25in]{geometry}
\usepackage{forest}
\usepackage{ulem}
\usepackage{colortbl}
\pagestyle{empty}

\begin{document}

\begin{tabularx}{\textwidth}{l|X}
\textbf{Quiz 1B} &   \textbf{Name: } \\
\hline
\end{tabularx}

\medskip

\begin{table}[h]
\centering
\arrayrulecolor{black!10} % Very light table outline
\begin{tabular}{|p{0.3\textwidth}|p{0.6\textwidth}|}
\hline
\begin{minipage}[t]{0.3\textwidth}
\vspace{0pt}  % This is crucial for top alignment
\begin{forest}
  for tree={
    font=\ttfamily,
    grow'=0,
    child anchor=west,
    parent anchor=south,
    anchor=west,
    calign=first,
    edge path={
      \noexpand\path [draw, \forestoption{edge}]
      (!u.south west) +(7.5pt,0) |- node[fill,inner sep=1.25pt] {} (.child anchor)\forestoption{edge label};
    },
    before typesetting nodes={
      if n=1
        {insert before={[,phantom]}}
        {}
    },
    fit=band,
    before computing xy={l=15pt},
  }
[\underline{/}
  [\underline{home}
    [\underline{maria}
      [.profile]
      [\underline{research}
        [experiment.log]
        [results.csv]
      ]
      [\underline{papers}
        [draft.txt]
        [outline.txt]
        [figures.pdf]
        [references.pdf]
      ]
      [\underline{archive}]
    ]
    [\underline{john}
      [\underline{bin}
        [install.sh]
      ]
      [\underline{temp}]
    ]
  ]
]
\end{forest}
\end{minipage}
&
\begin{minipage}[t]{0.55\textwidth}
\vspace{0pt}  % This is crucial for top alignment
\noindent \textbf{Instructions:} Please answer the following questions below making sure to write your answers legibly. If you run out of room or need to re-write please use the back of this quiz. Put your name on the line above.

\vspace{10pt}

The graphic on the left denotes the file system on a machine that we are using. We will use it to answer the questions below. A few important notes:
\begin{itemize}
    \item On the graphic \underline{directories} are underlined.
    \item Any other object should be considered a file.
    \item \textbf{For each question assume that you are starting from the original file system}. You should ignore any changes to the file system you made in previous questions.
    \item You will be graded for being unnecessarily complex in your solutions. 
    \item While you are welcome to use \texttt{bash} syntax we have not learned in class I will be testing any solution \emph{on my machine}. If it does not work or requires a non-standard package you will not receive credit.
    \item Unless otherwise stated all questions are worth the same number of points.
    \item All of the answers should be completed in a single line.
\end{itemize}
\end{minipage}
\\
\hline
\end{tabular}
\end{table}

\begin{enumerate}

\item Your current working directory is \texttt{/home/maria}. Please print the first ten lines in \texttt{research/experiment.log} using a relative path.
\vspace{1.25cm}

\item You are currently in the \texttt{/home/john} directory. Using a \emph{relative} path, list all the \texttt{.pdf} files which are in the \texttt{/home/maria/papers} directory.
\vspace{1.25cm}

\item Your current working directory is \texttt{/home/maria}. Please copy all \texttt{.pdf} files from the \texttt{papers} sub-directory to the \texttt{archive} directory. Use relative pathing.
\vspace{1.25cm}

\item Your current working directory is \texttt{/home/john}. The directory \texttt{/home/john/temp} has over 400 files in it (which is why they are not shown in the diagram). Please create a file \texttt{/home/john/temp\_list.txt} which contains a list of all files (including hidden) in the \texttt{/home/john/temp} directory.
\vspace{1.25cm}

\item Your current working directory is \texttt{/home/maria}. Please return all \emph{lines} in \texttt{papers/draft.txt} which have the word \texttt{results} in it. This should be a case-insensitive search.

\end{enumerate}


\end{document}