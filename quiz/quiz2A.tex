\documentclass[11pt]{article}
\usepackage{fullpage, fancyhdr, ifthen, amssymb, amsfonts, amsmath, booktabs}
\usepackage{tabularx}
\usepackage{multicol}
\usepackage[margin=0.25in]{geometry}
\usepackage{forest}
\usepackage{ulem}
\usepackage{colortbl}
\pagestyle{empty}

\begin{document}

\begin{tabularx}{\textwidth}{l|X}
\textbf{Quiz 2A} &   \textbf{Name: } \\
\hline
\end{tabularx}


\begin{table}[h]
\centering
\arrayrulecolor{black!10} % Very light table outline
\begin{tabular}{|p{\textwidth}|}
\hline
\begin{minipage}[t]{\textwidth}
\vspace{0pt}  % This is crucial for top alignment
\noindent \textbf{Instructions:} Please answer the following questions making sure to write your answers legibly. If you run out of room or need to re-write use the back of this quiz.  The graphic denotes the file system on a machine and the Dockerfile in \texttt{/users/nick/project}.
\begin{itemize}
    \item \underline{Directories} are underlined. Any other object should be considered a file.
    \item \textbf{For each question assume that you are starting from the original file system}. Ignore any changes to the file system you made in previous questions.
    \item You will be graded for being unnecessarily complex in your solutions. 
    \item Any syntax not covered in class will be tested \emph{on my machine} for applicability.

    \item \textbf{Do not assume a ``home" directory location.}
\end{itemize}


\end{minipage} \\
\hline
\begin{tabular}{p{0.45\textwidth}|p{0.55\textwidth}}
\begin{minipage}[t]{\linewidth}
\vspace{0pt}  % This is crucial for top alignment
\begin{forest}
  for tree={
    font=\ttfamily,
    grow'=0,
    child anchor=west,
    parent anchor=south,
    anchor=west,
    calign=first,
    edge path={
      \noexpand\path [draw, \forestoption{edge}]
      (!u.south west) +(7.5pt,0) |- node[fill,inner sep=1.25pt] {} (.child anchor)\forestoption{edge label};
    },
    before typesetting nodes={
      if n=1
        {insert before={[,phantom]}}
        {}
    },
    fit=band,
    before computing xy={l=15pt},
  }
[\underline{/}
  [\underline{users}
    [\underline{nick}
      [\underline{project}
        [Dockerfile]
      ]
      [\underline{old\_project}
        [\underline{jpgs}]
        [requirements.txt]]
    ]
  ]
]
\end{forest}
\end{minipage}
&
\begin{minipage}[t]{\linewidth}
\vspace{0pt}  % This is crucial for top alignment
\underline{\textbf{Dockerfile}}

\begin{verbatim}
FROM python:3.10.15-bookworm

WORKDIR /prj
COPY requirements.txt .
RUN pip install -r requirements.txt

RUN mkdir /prj/data

COPY jpgdir/*.jpg /prj/data

RUN echo $MY_NAME
\end{verbatim}
\end{minipage}
\end{tabular} \\
\hline
\end{tabular}
\end{table}


\noindent We are moving some things from our \texttt{old\_project} to the \texttt{project} directory. This Dockerfile does not build or run because of issues that we will fix in the questions. Out goal is to build an image, \texttt{quiz\_image} using the command \mbox{\texttt{docker build . -t quiz\_image}} from inside the \texttt{/users/nick/project} directory. 

\begin{enumerate}

\item Move \texttt{requirements.txt} to the correct location. Assume your current working directory is \texttt{/users/nick} and use a \emph{single command with relative} paths to move the \texttt{requirements.txt} file to the correct location. 
\vspace{1.5cm}
\item There are a number of \texttt{*.jpg} files in the \texttt{old\_project/jpgs}. Using \emph{absolute paths}, write a set of commands that will allow the \texttt{COPY} command in the Dockerfile to run properly.
\vspace{1.5cm}
\item The environment variable \texttt{MY\_NAME} will be printed on the terminal at build, however it is not currently set in the Dockerfile. Please write a line of code to be added to the Dockerfile to have it print ``NICK" when building.
\vspace{1.5cm} 
\item We run the container using \texttt{docker run -it quiz\_image /bin/bash}. Please modify this so that an environment variable (\texttt{MY\_VAR}) is set inside the container, with a value of \texttt{1234}. Write the entire \texttt{docker run} command.


\end{enumerate}


\end{document}